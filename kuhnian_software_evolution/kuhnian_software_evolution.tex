\documentclass[12pt,draft]{article}

\title{Kuhnian Software Evolution}
\author{Tom Wallis}
\date{}

\begin{document}
\maketitle

\begin{abstract}
  The process of developing complicated software systems is a research topic of
  its own merit, and as requirements for those systems change over time,
  maintenance of the system becomes unwieldy and complicated. Eventually, a
  ``tipping point'' comes about, and large swathes of the architecture are
  rewritten at great expense. Elements of this process are evident in literature
  on technical debt. This article explores the similarities between Thomas
  Kuhn's model of paradigm shift in philosophy of science and change in software
  architecture, justifying the analogy in that both software architectures and
  scientific paradigms are models of some aspect of the world and become
  increasingly inadequate metaphors for the subject they model over time. This
  is explored initially through a memoir of first-hand experience developing a
  complex software system with changing requirements, and then through an
  exploration of both software engineering and philosophy of science literature,
  making comparisons between the two and drawing insights from the comparisons.
\end{abstract}

\section{Introduction}

\end{document}