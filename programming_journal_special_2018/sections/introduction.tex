\section{Introduction}

\subsection{What's A Model?}
To understand a part of the world, we have to analyse it --- this can be done in
two ways. The first is to observe the world directly, making predictions to
verify a representation of what's known. The second is to take what's known, and
what others' predictions have shown to be true, and compose them together into a
trustworthy representation of what's known. Either way, we come to build a
representation of the world, and \emph{this} is our ``model''. \par

Systems modelling is an old field. There's an argument to be made that cave
paintings or rudimentary philosophical arguments constitute a model of a system
--- but this isn't helpful. Really, a model of a system is something we employ
to learn about the world, not \emph{just} to represent it. So, we're not \emph{just}
looking for a representation of the world: we're looking for a representation
which can be inspected to reveal something about its subject.\par

This review of literature in the field will cover a historical account of the
field, and examine a few of the different methods developed for building these
representations of the world, but comes to the conclusion that modern
developments simply aren't pragmatic for performing systems modelling in the
real world, due to difficulties in some key areas of modelling a system:

\begin{itemize}
\item Information capture
\item System analysis
\item Exploration of result
\end{itemize}

% \subsection{Competing Paradigms}
% What different ways of constructing these models have emerged?

% What are the competing philosophies around how models are constructed,
% populated, and analysed?
