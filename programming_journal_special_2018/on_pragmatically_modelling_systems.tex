\documentclass{tufte-handout}

\usepackage{indentfirst}
\usepackage{natbib}

\usepackage[textsize=small]{todonotes}

\newcommand{\teacher}[]{\emph{Teacher:} }
\newcommand{\student}[]{\emph{Student:} }

\bibliographystyle{unsrtnat}

\title{On Pragmatically Modelling Systems}
\author{Wallis, William \& Storer, Tim}
\date{}

\begin{document}
\maketitle

\begin{abstract}
  System modelling is increasingly important as a method by which we can make
  sense of the world and improve it. To model a system is to take some construct
  and build a representation of it --- but many philosophies on how this should
  be done prevail, each with their own successes and shortcomings. Aside from
  being important, systems modelling is also deeply interdisciplinary, which has
  the consequence of fracturing knowledge on the topic and effort toward it.
  Here, the current state of systems modelling across a few fields is compiled,
  and examined for their practicality in real-world systems modelling. From
  these observartions, an autopsy of the field is presented, and some avenues
  forward are suggested.
\end{abstract}


% As this stands, I think I want to make this _effectively_ the lit review for
% the PhD. It should cover lots of the literature I've read, but it should also
% contain insights (the sorts of things Tim points out that SLRs _don't_
% typically contain, that this can, because of the special issue's requirements.)

% An essay on the current state of systems modelling:
%   - The progress we've made
%   - The problems we're yet to solve
%   - Where we would ideally get to
%   - Some potential pitfalls
%     - Building something pragmatic
%     - Dealing with Tim's three inconsistencies

\section{Introduction}

% Make more contextual reference to the dojo being some sort of holy retreat, as
% if it's some sort of religion for programmers a-la thecodelesscode.

\footnotesize{\emph{In a mountain retreat for programmers, a student visits
    hoping to gain wisdom about how to avoid critical systems failing at
    essential moments.}}

\student Teacher, the world is strange, and trying to produce reliable software
in it can be quite difficult. I can't imagine how people make sense of
complicated events that overturn our meticulous engineering! I've heard of flash
crashes on stock markets, of unreliable voting systems, and of medical systems
failing to provide vital care to those who need it. What wisdom do you have for
avoiding these mistakes, when we build important systems in an unpredictable world?

\teacher It's not useful to answer your question before we know the nature of
the world. To develop software without knowing its environment is to buy a chair
without seeing its house: if it fulfils its function, it will do so
ill-fittingly.
\noindent The world is built on a series of complicated systems, interacting and
influencing each other. In early times, eager students of many fields set aside
their differences to build the General Systems Theory, a method of modelling the
world's many facets with a common method for all students\ldots{}~but it was not
enough.
\noindent General Systems Theory placed a focus on how the world related
together, framing all things as either entities or relations between them. For a
time, this was enough, but the early theorists developed their framework with
little regard for a system's \emph{behaviour}: how their structures might change
through time. Many solutions were suggested: Cybernetics, Cellular Automata, and
many new system theories. Arguably, the most successful was Chaos Theory, but
---

\student --- but we have had chaos theory for decades! Is even the most
successful theory of systems --- including their behaviours --- still not
enough?

\teacher We have to be patient. The third act takes place now, in research into
\emph{Complex Adaptive Systems}. These systems change both their structure and
their behaviour in response to a range of inputs, allowing the model to evolve
in profound ways that we still struggle to understand. To date, these Complex
Adaptive Systems remain elusive to study, due to their highly dynamic nature, as
well as the difficulties in capturing all of the relevant information in a model
so as to accurately simulate your ``unpredictable world''.
\noindent If we can properly simulate a complex adaptive system, we can better
understand the environment of your vital systems, and perhaps better prepare for
catastrophic events; or better yet, adapt our systems so as to mitigate these
events by reacting and responding to them.

\student If these system models are so powerful, why don't we already use them
to investigate the world?!

\teacher That's because as our models grow larger, they become intractibly
difficult to build accurately; a model of a CAS must capture all of a system's
state, behaviour, \emph{and} the variance present in each as the system responds
to its environment. To build these models at scale is a challenge we are still
solving.

\student Perhaps I can help: in a recent sprint a lead developer insisted that
our team begin employing Aspect Oriented Programming. Aspect Orientation allowed
us to separate some orthoganal concerns of our code, such as debugging and
logging, from the core business logic. Along similar lines, if we can construct
models of non-adaptive systems at scale, and apply the variation separately to
the simulation of the problem domain itself, then the core simulation doesn't
have to deal with a CAS' additional detail; it can appear as a regular
simulation, and variation to agents' behaviour could be applied separately.

\teacher Aspect orientation appears little in the simulation of agent behaviour.
This avenue hasn't been explored much yet; perhaps it can offer a way forward in
building practical models of complicated systems.

\student That's not all! Using aspects for these orthoganal concerns to the
actual problem domain, experiments and alternative simulations could be carried
out not by editing an unwieldy and complex model, but by swapping one set of
aspects for another. Then, experiments and analyses on a problem can be
represented by a small implementation of the problem domain, and a library of
aspects which compose the system itself.

\teacher Can you be certain that such a method can really capture the myriad
dimensions of a complex adaptive system? Can such a method be as accurate as a
hand-crafted simulation with adaptation hardcoded into the problem domain?

\student If aspect orientation really does appear little in system modelling
literature, then the only way to know is to try. I will return once I have
conducted a Research Project, and will present my meditations on my findings.

\indent \emph{The teacher nods, turns, and returns to their patient study, a
  signal to their student; the enthusiastic new recruit parts ways and exits the
  dojo, making a mental note to prepare the rites of the Literature Review.}




% \subsection{What's A Model?}
% To understand a part of the world, we have to analyse it --- this can be done in
% two ways. The first is to observe the world directly, making predictions to
% verify a representation of what's known. The second is to take what's known, and
% what others' predictions have shown to be true, and compose them together into a
% trustworthy representation of what's known. Either way, we come to build a
% representation of the world, and \emph{this} is our ``model''. \par

% Systems modelling is an old field. There's an argument to be made that cave
% paintings or rudimentary philosophical arguments constitute a model of a system
% --- but this isn't helpful. Really, a model of a system is something we employ
% to learn about the world, not \emph{just} to represent it. So, we're not \emph{just}
% looking for a representation of the world: we're looking for a representation
% which can be inspected to reveal something about its subject.\par

% This review of literature in the field will cover a historical account of the
% field, and examine a few of the different methods developed for building these
% representations of the world, but comes to the conclusion that modern
% developments simply aren't pragmatic for performing systems modelling in the
% real world, due to difficulties in some key areas of modelling a system:

% \begin{itemize}
% \item Information capture
% \item System analysis
% \item Exploration of result
% \end{itemize}

% Ultimately, a question is raised by the literatue as it currently stands. Is
% there necessarily a tradeoff involved between these 

% \subsection{Competing Paradigms}
% What different ways of constructing these models have emerged?

% What are the competing philosophies around how models are constructed,
% populated, and analysed?

\section{An Overview of Paradigms}\label{sec:paradigms}
\todo{Change filename to reflect new section header}~Several paradigms have been
produced with different philosophies around how to approach systems modelling.
Some approaches are deeply mathematical in nature, These formal methods have
their own distinct flavours.

Some approaches are logical in nature\todo{citations}, which allows users of
these methods to assert specific truths and falsities about the model at hand.
\include{sections/contrasting_approaches}
\include{sections/aetiology_of_our_state}
\section{The Way Forward}

\subsection{Aetiology Of Our Issues}
\section{Conclusion}
Well, we certainly all learned lots here today. Thanks for reading \&{}c\&{}c\&{}c\ldots{}


\bibliography{lib}
\end{document}