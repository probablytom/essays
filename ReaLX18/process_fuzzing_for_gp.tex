\documentclass[12pt]{llncs} % TODO: change this to Springer format

\author{Tom Wallis \& Tim Storer}
\institute{\email{w.wallis.1@research.gla.ac.uk} \& \email{timothy.storer@glasgow.ac.uk} \\ School of Computing Science, University of Glasgow
}
\date{}
\title{Process Fuzzing as an Approach for Genetic Programming}

\usepackage[obeyDraft,textsize=tiny]{todonotes}
\usepackage{hyperref}
\usepackage{cleveref}
\usepackage[final]{listings}  % final gives us listings even with the document is in draft.

\bibliographystyle{splncs04}

\lstset{language=python}
\lstset{basicstyle=\ttfamily\small}
\lstset{keywordstyle=\ttfamily\bfseries}
\lstset{frame=single}


\begin{document}

\maketitle

\begin{abstract}
  Genetic Programming (GP) has recently seen a growing application in the area
  of writing and improving computer programs. Generally, for experiments in this
  area, bespoke tools are constructed to perform research. In this paper, it is
  demonstrated that GP behaviour can be achieved via \emph{process fuzzing}, and
  an implementation of the adaptation of ASTs for GP behaviour in the process
  fuzzing tool PyDySoFu is described.
\end{abstract}


\section{Evolving Programs} Genetic Programming (GP) is a very well-established
and promising field which has returned impressive results in a number of areas.
The field has spawned a number of similar approaches which, while based on the
same underlying concept, attempt an evolution-based solution in novel ways.
Cartesian GP\cite{miller2011cartesian} uses a directed graph to represent a
solution to a problem, and has seen great success, for its ability to converge
on an acceptable solution in a relatively short number of generations.
Stack-based GP\cite{perkis1994stack} employs the use of multiple different
stacks so as to work with state and keep track of multiple values, which can be
difficult for the traditional tree-based genetic programming (TGP) approach.\par

\subsection{Genetic Improvement}
Generally, variants of GP present methods for improving mathematical-looking
functions against some fitness function. However, variants have begun to arise
which, rather than mutating some abstract representation of a program, mutate
the program itself. Stack-based GP in the Push family of
languages\cite{spector2001autoconstructive}, for example, features an approach
where values on stacks can be code, which can be subsequently executed; in this
way, Stack-based GP can be used to achieve a kind of metaprogramming. Similarly,
Linear Genetic Programming\cite{brameier2007linear} (LGP) is a method which
evolves a series of instructions, rather than a tree of operations, to achieve a
solution.\par

Indeed, approaches involving the alteration of source code have garnered a
growing amount of attention: in the improvement of Java programs alone, several
tools for the improvement of a codebase have
arisen\cite{cody2015locogp,arcuri2008multi,castle2012evolving}. As well as
improving codebases, genetic improvement-style metaprogramming could be used to
implement solutions to problems in GP, by constructing imperative processes that
fit a curve, rather than a functional-style tree representation as in
TGP\cite{koza1994genetic}.\par



\section{Approaches with Process Fuzzing}

\subsection{A Brief Note on Process Fuzzing}
Ultimately, Genetic Programming relies on the mutation and evaluation of a
representation of a problem's solution. Process fuzzing allows for this to be
achieved for imperative code, by modifying and rewriting it prior to execution.
A tool implementing this is PyDySoFu\cite{pydysofu}. PyDySoFu catches function
calls and --- every time a function is executed --- modifies the function's
AST\footnote{An Abstract Syntax Tree is a tree representation of an expression
  in a language with a formal grammar, such as programming languages.}
and runs the resulting code, rather than the original. This modification is
performed by passing the AST through a particular function, called a ``process
fuzzer'' (or ``fuzzer'').\par

There is a clear link between the requirement for mutation in GP and the
functionality provided by a fuzzer. Some work is required, however, to represent
GP-like interactions within the tool. Specifically:

\begin{enumerate}
\item Multiple variants must be generated and their outputs tracked, so they can
be compared to each other, and ranked.
\item This ranking must be done by some function appropriate for the problem
domain at hand --- GP's ``fitness functions''.
\item Once variants are ranked, it must be possible to recombine successful ones
and use these in future generations.
\end{enumerate}



\subsection{Improving PyDySoFu}
PyDySoFu was originally unable to keep track of multiple variants, nor record
the return values of the variants it produced. The solution was to extend
PyDySoFu's underlying mechanism into a more fully featured aspect orientation
framework, capable of more sophisticated code weaving.\par

This extension became an aspect orientation framework, ASP\cite{asp}. ASP's
pertinent feature is its abilty to weave advice around a method of a class, such
that functions can be executed before and after the method is called, without
being coupled to the original codebase.\par 

\subsection{Implementing GP-Like Behaviour}
Critically, aspects in ASP can be objects. When fuzzers are written as instances
of aspect classes, they can use instance variables to keep track of the variants
they generate between invocations. Also, because ASP is capable not only of
including behaviour before method invocation, but also after, PyDySoFu can
utilise this to catch the output from the method call, and use this to rank
variants. This satisfies the first of the three earlier criteria.\par

When instances of fuzzers are created, a success metric\footnote{Improvements to
  PyDySoFu were not originally developed with GP specifically in mind.
  Therefore, generations are referred to as ``rounds'', and fitness functions as
  ``success metrics''.} can be passed to its constructor, and this is
kept within the object's state --- this can then be used in the ranking of
variants in a round, fulfilling the second of the above criteria.\par

Recombination of variants can be done by combining modified ASTs from variants
in the previous round when constructing a new one --- this fulfils the third of
the above criteria, and is implemented in such a way as to make recombination
easily tailored to individual problem domains via subclasses. Source for the
\texttt{GeneticImprover} implementation of these improvements can be found in
the project's repository\cite{pydysofu}.\par

\subsection{Benefits of the Approach}
Use of PyDySoFu as a GP solution has a number of advantages. First, it is under
active development, meaning that the tool can be expected to improve. Users can
anticipate PyDySoFu to be a fertile ground for new research, where process
fuzzing can be used to separate concerns in a variety of fields.\par

Importantly, PyDySoFu is not just a tool for implementing solutions to GP
problems. Its versatility is a second benefit: its most active area of study,
socio-technical variance, provides a plethora of problem domains where GP might
find applications. Performing this research without a cross-disciplinary tool
would require lots of ancillary work, but PyDySoFu bridges this gap.\par

Further, PyDySoFu is able to fuzz code \emph{as it is run} (``dynamic
fuzzing''). This functionality can be used to perform experiments with GP
solutions which might --- for example --- use dynamic fuzzing to represent
solutions which operate in an unreliable real world, such as an unreliable
network or anomalies in animal populations.\par


\section{Future Work}
\label{sec:future_work}
PyDySoFu is a new entrant into tools for running experiments within GP, with the
unusual trait that its suitability for evaluating GP problems comes from its
versatility, meaning that PyDySoFu is positioned to be an unusually effective
tool in a variety of fields. Many things can be done to increase PyDySoFu's
effectiveness as a GP tool, and to exploit it's versitility to explore new
research possibilities, including:

\begin{itemize}

\item A wider array of GP-style fuzzers can be implemented, building on the
  broad array of code-improving GP approaches surfacing in the literature.
  These could also be used to replicate previous work in the field.
  
\item Further exploration of GP using AST-style program mutation for codebase
  improvement can also be explored in Python using PyDySoFu, which, combined
  with the other research opportunities, makes it an exciting alternative to
  existing solutions.

\item Given PyDySoFu naturally links socio-technical modelling and GP,
  experiments involving GP solutions to socio-technical problems are now
  feasible. A major contribution of GP interactions in PyDySoFu is that the
  tool's versitility allows GP to be explored within socio-technical problem
  domains\cite{wallis2018a}.

\end{itemize}


\section{Conclusion}
This paper has given a brief overview of recent development of PyDySoFu, a
process fuzzing tool which is now capable of GP-style interactions. While
GP-style interactions arising from process fuzzing is an interesting result of
its own, the availability of the tool should inspire further genetic
metaprogramming work, and encourage researchers to make use of its
potential across a variety of domains.\par

\section*{Acknowledgements}
\label{sec:ack}
The authors would like to thank Obashi Technology for helping to fund this
research, and Rob Dekkers for his help reviewing this work.


\bibliography{lib}
\end{document}